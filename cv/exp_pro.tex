%-------------------------------------------------------------------------------
%	SECTION TITLE
%-------------------------------------------------------------------------------
\cvsection{Expériences professionnelles}


%-------------------------------------------------------------------------------
%	CONTENT
%-------------------------------------------------------------------------------
\begin{cventries}

%---------------------------------------------------------
\cventry
  {Chercheur} % Affiliation/role
  {Ecole Polytechnique de Dakar, CIRAD} % title
  {UCAD Dakar, SNG} % Location
  {Depuis 2021} % Date(s)
  {
    \begin{cvitems} % Description(s) of experience/contributions/knowledge
      Accueilli à l'Ecole polytechnique de Dakar, et au sein de l'unité mixte internationale de l'IRD UMMISCO (Unité Modélisation Mathématique et Informatique des Systèmes COmplexe). Je poursuis mon projet scientifique en investiguant la relation entre l'approche par les communs et la théorie de la viabilité.
    \end{cvitems}
  }


  %---------------------------------------------------------
  \cventry
    {Chercheur} % Affiliation/role
    {UMR SENS, CIRAD} % title
    {CIRAD - Montpellier, FR} % Location
    {Depuis 2018} % Date(s)
    {
      \begin{cvitems} % Description(s) of experience/contributions/knowledge
        A l'UMR SENS en tant que géographe et modélisateur, je travaille sur l’émergence de la coopération et de l’entraide. Je considère l’espace comme une production sociale, et celle‑ci est robuste aux changements d’échelle. En entrant par les Communs de la terre et des ressources qu’elle porte, je considère les Communs comme un proxi permettant d’identifier spatialement les situations d’entraide.
      \end{cvitems}
    }


%---------------------------------------------------------
\cventry
  {Post-doctorat} % Affiliation/role
  {OHMi Téssékéré} % title
  {CNRS - Dakar, SNG} % Location
  {2017-2018} % Date(s)
  {
    \begin{cvitems} % Description(s) of experience/contributions/knowledge
      A l’OHMi Téssékéré (Observatoire Hommes-Milieux International Tessékéré, Sénégal) : développement d’outils d’aide à la décision et à la médiation pour les acteurs impliqués dans le projet de la Grande muraille verte. Ce poste s’intègre au projet ANR Future‑Sahel et est basé à Dakar. Contrat de 12 mois.
    \end{cvitems}
  }

%---------------------------------------------------------
\cventry
  {Post-doctorat} % Affiliation/role
  {Chaire d'excellence "Capital environnemental et gestion durable des cours d'eau"} % title
  {Fondation de l'université de Limoges, FR} % Location
  {2015-2017} % Date(s)
  {
    \begin{cvitems} % Description(s) of experience/contributions/knowledge
      Projet de recherche autour de l'émergence de la coopération basée sur les collectifs d'irrigants dans le département des Pyrénées-Orientales. Contrat de 16 mois.
    \end{cvitems}
  }

%---------------------------------------------------------
\cventry
  {Ingénieur d'étude} % Affiliation/role
  {Pôle d'appui à la recherche} % title
  {Université de Limoges, FR} % Location
  {2015} % Date(s)
  {
    \begin{cvitems} % Description(s) of experience/contributions/knowledge
    Mise en place de l'infrastructure informatique pour proposer de la cartographie dynamique pour l'université, basée sur Debian, Mapserver, MySQL, Openlayer3 et Leaflet. Contrat de 6 mois.
    \end{cvitems}
  }

%---------------------------------------------------------
\cventry
  {Ingénieur d'étude} % Affiliation/role
  {Fondazion E. Mach} % title
  {San Michele all'Adige, IT} % Location
  {2011} % Date(s)
  {
    \begin{cvitems} % Description(s) of experience/contributions/knowledge
    Gestion complète d'un projet, de l'étude d'avant-projet à la mise en place du prototype d'infrastructure de serveurs cartographiques. Travail sur des données LIDAR pour la détection de formes viticoles à l'échelle du Trentino (Italie). Contrat de 8 mois.
    \end{cvitems}
  }

%---------------------------------------------------------

\end{cventries}
