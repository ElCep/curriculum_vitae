%-------------------------------------------------------------------------------
%	SECTION TITLE
%-------------------------------------------------------------------------------
\cvsection{Projets}


%-------------------------------------------------------------------------------
%	CONTENT
%-------------------------------------------------------------------------------
\begin{cventries}

%---------------------------------------------------------
\cventry
{S. Martin} % leaders
{LifeLine} % project name
{INRAe, Paris, France} % Location
{2025 -- 2028} % Date(s)
{
  \begin{cvitems} % Description(s) of tasks/responsibilities
    Ce projet fait partie de la grappe de projet PEPR Math-vives, et a pour objectif de développer une nouvelles approche de la théorie de la viabilité intégrants une approche "agent". \\
    \textbf{Ma responsabilité } : Montage du projet, suivie et animation au Sénégal autout du projet Santés et Territoire.
  \end{cvitems}
}  

  %---------------------------------------------------------
\cventry
{I. Alvarez} % leaders
{Vallium} % project name
{INRAe, Paris, France} % Location
{2024 -- 2025} % Date(s)
{
  \begin{cvitems} % Description(s) of tasks/responsibilities
    Ce projet fait partie de l'initiative TSARA (Transformer les Systèmes Alimentaires et l’Agriculture par la Recherche en partenariat avec l’Afrique), et a pour objectif de tester l'opérationnalité de la théorie de la Viabilité dans le cadre de projet de developpement. \\
    \textbf{Ma responsabilité } : Montage du projet, suivie et animation au Sénégal autout du projet Santés et Territoire.
  \end{cvitems}
}  
%---------------------------------------------------------
\cventry
{J.-L. Chotte} % leaders
{FerloSine} % project name
{IRD, Dakar, Sénégal} % Location
{2023 -- 2026} % Date(s)
{
  \begin{cvitems} % Description(s) of tasks/responsibilities
    Ce projet fait partie de la grappe PEPR SLAM-B. Il a pour ambitions de faire la preuve du concept de cette relation locale-globale entre neutralité carbone et neutralité de la dégradation des terres et objectifs de développement durable pour promouvoir un agro-sylvo-pastoralisme durable au Sahel. \\
    \textbf{Ma responsabilité } : Suivi du développement de l'outil de modélisation d'évaluation intégrée (MEI) permet de relier les principales caractéristiques de la société et de l'économie avec la biosphère et l'atmosphère dans un cadre de modélisation unique.
  \end{cvitems}
}

%---------------------------------------------------------
\cventry
{C. Piou} % leaders
{Projet AFD-CLCPRO} % project name
{CIRAD, Montpellier, FR \hspace{5em} CNLA Mauritanie} % Location
{2022 -- 2024} % Date(s)
{
  \begin{cvitems} % Description(s) of tasks/responsibilities
    Ce projet financé par l'AFD en partenariat avec la CLCPRO, a pour objectif de consolider les bases de la stratégie de lutte préventive et développer la recherche opérationnelle sur le Criquet pèlerin en région occidentale. Il s'agit donc d'apporter un soutient scientifique aux méthodes de lutte autour du développement de nouveaux modèles.\\
    \textbf{Ma responsabilité } : responsable de l'activité 2.2.2 "co-construction de modèles pour identifier les critères d’intervention rapide" au sein de l'unité. J'ai eu la charge de recruter et co-encadrer 3 étudiants de M2. J'ai développé une méthodologie de co-construction de décisions sur la base de simulation bayésienne pour mettre en lumière le processus de lutte anti-acridienne.
  \end{cvitems}
}

%---------------------------------------------------------
\cventry
{R. Duboz} % leaders
{Projet Santés \& Térritoires} % project name
{CIRAD, Montpellier, FR \hspace{5em} Sénégal, Bénin} % Location
{2022 -- 2025} % Date(s)
{
  \begin{cvitems} % Description(s) of tasks/responsibilities
    Santé \& Territoires vise à concevoir, tester et évaluer une démarche participative et inclusive innovante pour accompagner la transition agroécologique. L’ambition est d’améliorer conjointement la santé des humains, des systèmes de production agricole et de l’environnement, puis d’impacter positivement et durablement les moyens d’existence des populations sur des territoires donnés.\\
    \textbf{Ma responsabilité } : J'ai pris en charge les aspects de modélisation d'accompagnement au Sénégal et au Bénin en appuis au chef de projet. Au Sénégal, j'ai co-encadré 2 stagiaires ingénieur sur le développement d'interfaces pour la plateforme du projet, et 2 étudiants en stage d'ingénieur sur le prototypage de station autonome d'analyse de qualité de l'eau. J'ai par ailleurs pris en charge une activité de co-construction de modèles avec un groupe de pêcheurs autour du lac. Au Bénin, je suis intervenu 3 fois sur de l'accompagnement des équipes locales pour mettre en place des approches de modélisation d'accompagnement.
  \end{cvitems}
}

  %---------------------------------------------------------
  \cventry
    {P. Salgado et M.F. Diong} % leaders
    {Projet CaSSECS} % project name
    {CIRAD, Montpellier, FR \hspace{5em} ISRA, Dakar, SNG} % Location
    {2020 -- 2023} % Date(s)
    {
      \begin{cvitems} % Description(s) of tasks/responsibilities
        Le projet CaSSECS (\emph{Carbon Sequestration and GHG emissions in (agro)Sylvopastoral Ecosystems in the sahelian CILSS States}), financé par le programme européen DeSIRA a pour objectif global de fournir aux pays de l’espace CILSS des connaissances sur les facteurs d’émissions de gaz à effet de serre et des données de référence. Avec ces éléments, ces pays pourront mieux établir le bilan carbone saisonnier et annuel des écosystèmes agro‑sylvopastoraux. Ce bilan sera une base du plaidoyer dans les négociations internationales quant aux engagements pris dans le cadre de l’accord de Paris sur le climat.\\
        \textbf{Ma responsabilité } : responsable de la tâche ”Modélisation d’accompagnement”, pilotage des taches autour de l'accompagnement dans les activité de reforestation. Co‑encadrement d’un doctorant.
      \end{cvitems}
    }

%---------------------------------------------------------
  \cventry
  {C. Jahel} % leaders
  {Prospective Niayes 2040} % project name
  {CIRAD, Montpellier, FR} % Location
  {2019 -- 2021} % Date(s)
  {
    \begin{cvitems} % Description(s) of tasks/responsibilities
      Ce projet est à considérer comme un ”\textit{proof of concept}”, visant à accompagner des acteurs locaux (ici dans la zone des Niayes au Sénégal) d’un questionnement prospectif au passage à l’action en mobilisant les outils de l’anticipation et en les couplant avec l’approche ComMod.\\
      \textbf{Ma responsabilité} : guider le développement d'un dispositif expérimental autour du jeu pour répondre aux objectifs sociaux. Mise en production du jeu et acquisition des résultats.
    \end{cvitems}
  }

%---------------------------------------------------------
  \cventry
  {J.-P. Venot} % leaders
  {Projet ANR DouBT} % project name
  {IRD, Montpellier, FR} % Location
  {2018 -- 2019} % Date(s)
  {
    \begin{cvitems} % Description(s) of tasks/responsibilities
      Le Projet DouBT (\emph{Deltas' Dealings with Uncertainty}), financé par l'Agence nationale de la recherche (France, ANR‑15‑ORAR‑0002) propose d'apporter un éclairage novateur sur les façons de conceptualiser et de gérer l'incertitude dans le secteur de la planification environnementale des deltas d'Asie du Sud-Est.\\
      \textbf{Ma responsabilité} : conception d'un jeu co-construit avec les acteurs locaux et les partenaires.  Supervision des ateliers avec les acteurs. Encadrement d’un étudiant en stage de Master 2 sur les incertitudes.
    \end{cvitems}
  }

%---------------------------------------------------------
  \cventry
  {N. Gadet et A. Caron} % leaders
  {Projet ProSuLi} % project name
  {CIRAD, Montpellier, FR} % Location
  {2018 -- 2022} % Date(s)
  {
    \begin{cvitems} % Description(s) of tasks/responsibilities
      Le projet européen ProSuLi (\emph{Promoting Sustainable Livelihoods in TFCAs}, Grant FED/2017/394-443), se donne pour ambition d'explorer les possibilités de gestion collective des \emph{TFCAs} (\emph{Transfrontier Conservation Areas}). Une démarche d’accompagnement des acteurs locaux est mise en place dans quatre sites (quatre pays d’Afrique australe).\\
      \textbf{Ma responsabilité} : sur le terrain du Botswana, expertise et animation d’ateliers de co‑construction de scénarii prospectifs. Élaboration d'un jeu co-construit avec les acteurs. Animation d'ateliers et acquisition de résultats pour aider la décision des orientations stratégiques.
    \end{cvitems}
  }

%---------------------------------------------------------
  \cventry
  {D. Masse et A. Falot} % leaders
  {Projet DSCATT} % project name
  {CIRAD, Montpellier, FR} % Location
  {2018 -- 2023} % Date(s)
  {
    \begin{cvitems} % Description(s) of tasks/responsibilities
        Le projet DSCATT (\emph{Dynamics of Soil Carbon Sequestration in Tropical and Temperate Agricultural systems}) est financé par Agropolis Fondation et la Fondation Total. Les approches interdisciplinaires mises en place ont pour objectif d’investiguer les possibilités de mobiliser le stockage du carbone dans les sols, à travers quatres sites (1 en France, 3 en Afrique : Sénégal, Zimbabwe, Kenya).\\
        \textbf{Ma responsabilité} : animateur du WP3-Territoire et co-annimateur du WP4 "modélisation d'accompagnement". Développement d’une méthodologie d’explicitation des savoirs. Encadrement de deux étudiants en stage de Master 2. Encadrement d’un post‑doc.
    \end{cvitems}
  }

%---------------------------------------------------------
  \cventry
  {S. Aubert} % leaders
  {Chantier "opérationnaliser les Communs"} % project name
  {CIRAD, Montpellier, FR} % Location
  {2018 -- 2021} % Date(s)
  {
    \begin{cvitems} % Description(s) of tasks/responsibilities
      Ce travail s’inscrit dans la continuité des activités du Comité Technique « Foncier et développement » de la Coopération française. L’objectif est de proposer à l’AFD (Agence française de développement) des outils permettant de prendre en compte les Communs et d’en favoriser l’approche dans les projets de développement.\\
      \textbf{Ma responsabilité} : expertise auprès de l'AFD, rédaction d'un guide à destination des agents de développement.
    \end{cvitems}
  }

%---------------------------------------------------------
  \cventry
  {D. Goffner} % leaders
  {Projet ANR Future-Sahel} % project name
  {UMI 3189 CNRS, Dakar, SNG} % Location
  {2017 -- 2019} % Date(s)
  {
    \begin{cvitems} % Description(s) of tasks/responsibilities
    Le projet Future-Sahel, financé par l’Agence nationale de la recherche (France, ANR‑15‑CE03‑0001), a pour objectif de fournir une aide scientifique et technique à l’Agence nationale de la Grande muraille verte au Sénégal.\\
    \textbf{Ma responsabilité} : construction d’une base de données spatiales, guider l'élaboration de méthodologies de quantification des usages autour de l’arbre dans le Sahel, animation d’ateliers.
    \end{cvitems}
  }

%---------------------------------------------------------
  \cventry
  {M. Chevallier et E. Delay} % leaders
  {Programme COMMONS} % project name
  {UMR 6042 CNRS, Limoges, FR} % Location
  {2016} % Date(s)
  {
    \begin{cvitems} % Description(s) of tasks/responsibilities
      Le programme \textsc{Commons} "\textit{COllective ManageMent Of Natural reSources}" a été financé par l’Appel d’offres à dimension internationale (AOI) de l’Université de Limoges et par la Fondation partenariale de l’université. Objectif : structurer un réseau de partenaires internationaux pour engager une approche interdisciplinaire autour des processus de gestion collective des ressources naturelles.\\
      \textbf{Ma responsabilité} : coordination d'un séminaire international autour des questions des Communs (Grèce, Maroc, Portugal, France).
    \end{cvitems}
  }

%---------------------------------------------------------
  \cventry
  {N. Becu et M. Amalric} % leaders
  {Projet LittoSim} % project name
  {UMR 7266 CNRS, La Rochelle, FR} % Location
  {2015 -- 2018} % Date(s)
  {
    \begin{cvitems} % Description(s) of tasks/responsibilities
      Le projet LittoSim a été financé dans le cadre du "Défi littoral 2015" du CNRS et a rassemblé 11 chercheurs français. Objectif : développer un outil pédagogique permettant de sensibiliser les aménageurs du territoire aux mesures de prévention et d'aménagement liées au risque de submersion marine.\\
      \textbf{Ma responsabilité} : développement de briques logiciels, de méthodes de débriefing et d’outils de visualisation.
    \end{cvitems}
      %\begin{cvsubentries}
      %  \cvsubentry{}{KNOX(Solution for Enterprise Mobile Security) Penetration Testing}{Sep. 2013}{}
      %  \cvsubentry{}{Smart TV Penetration Testing}{Mar. 2011 - Oct. 2011}{}
      %\end{cvsubentries}
    }

%---------------------------------------------------------

\cventry
{S. Leturcq et A. Lammoglia} % leaders
{Programme VitiTerroir} % project name
{UMR 7324 CNRS, Tours, FR} % Location
{2014 -- 2016} % Date(s)
{
  \begin{cvitems} % Description(s) of tasks/responsibilities
    Le programme VitiTerroir a mis au point un outil prospectif fondé sur une modélisation des transformations des terroirs viticoles dans la longue durée.\\
    \textbf{Ma responsabilité} : co-animation de l'axe modélisation spatiale et temporelle du projet.
  \end{cvitems}
}

%---------------------------------------------------------
\cventry
{N. Ollat et J-M. Touzard} % leaders
{Programme LACCAVE} % project name
{INRAE, FR} % Location
{2012 -- 2015} % Date(s)
{
  \begin{cvitems} % Description(s) of tasks/responsibilities
  Le programme LACCAVE (\emph{Long term impacts and Adaptation to Climate ChAnge in Viticulture and Enology}) a rassemblé 23 laboratoires de recherche en France, pour évaluer les effets du changement climatique sur la vigne et explorer les stratégies d'adaptation mobilisables. Il se poursuit aujourd'hui dans LACCAVE 2.0.\\
  \textbf{Ma responsabilité} : organisation d'un séminaire sur le  couplage de modèles, organisation d'une journée d'étude Changement climatique et viticulture. Animation scientifique.
  \end{cvitems}
}

%---------------------------------------------------------
\cventry
{M. Pontalti et F. Zottele} % leaders
{Ricerca sulla viticoltura di montagna} % project name
{Fondazione E. Mach, San Michele all'Adige, IT} % Location
{2011 -- 2018} % Date(s)
{
  \begin{cvitems} % Description(s) of tasks/responsibilities
    Études et expérimentations pour le maintien d'une viticulture de montagne dans le Trentino et plus largement sur des problématiques européennes, en partenariat avec le CERVIM (\emph{Centre d'études et de recherche sur la viticulture de montagne}).\\
    \textbf{Ma responsabilité} : développement d’un \emph{workflow} de l’évaluation des paysages de terrasses et de viticulture dite « héroïque ».
  \end{cvitems}
}

%---------------------------------------------------------
\cventry
{H. Quénol} % leaders
{Projet ANR TerViClim} % project name
{UMR 6554 CNRS, Rennes, FR} % Location
{2011 -- 2015} % Date(s)
{
  \begin{cvitems} % Description(s) of tasks/responsibilities
    Le projet TerViClim, d’observation et spatialisation du climat des terroirs viticoles mondiaux dans un contexte de changement climatique, financé par l’Agence nationale de la recherche (France, ANR‑07‑JCJC‑0069), a notamment permis l’installation de capteurs de température sur l’AOC Banyuls‑Collioure (Pyrénées-Orientales, FR) et dans le Val di Cembra (Trentino, IT).\\
    \textbf{Ma responsabilité } : modélisation spatiale des dynamiques de température à l'échelle fine sur les vignobles de l'AOC Banyuls et du Val di Cembra.
  \end{cvitems}
}

%---------------------------------------------------------

\end{cventries}
