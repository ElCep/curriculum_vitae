%-------------------------------------------------------------------------------
%	SECTION TITLE
%-------------------------------------------------------------------------------
\cvsection{Projects}


%-------------------------------------------------------------------------------
%	CONTENT
%-------------------------------------------------------------------------------
\begin{cventries}

%---------------------------------------------------------
  \cventry
    {P. Salgado et M.F. Diong} % leaders
    {CaSSECS project} % project name
    {CIRAD, Montpellier, FR \hspace{5em} ISRA, Dakar, SNG} % Location
    {2020 -- 2023} % Date(s)
    {
      \begin{cvitems} % Description(s) of tasks/responsibilities
        CaSSECS (\emph{Carbon Sequestration and GHG emissions in (agro)Sylvopastoral Ecosystems in the sahelian CILSS States}), funded by the European DeSIRA program, has the overall objective of providing to the CILSS countries, knowledge on greenhouse gas emission factors and reference data. With these elements, these countries will be able to better establish the seasonal and annual carbon balance of agro-sylvopastoral ecosystems. This assessment will be a basis for advocacy in international negotiations regarding commitments made in the Paris climate agreement.\\
        \textbf{My responsibility} : responsible for the task "Companion Modelling (ComMod)", piloting tasks around the accompaniment in reforestation activities. Co-supervision of a PhD student.
      \end{cvitems}
    }

%---------------------------------------------------------
  \cventry
  {C. Jahel} % leaders
  {Prospective Niayes 2040} % project name
  {CIRAD, Montpellier, FR} % Location
  {2019 -- 2021} % Date(s)
  {
    \begin{cvitems} % Description(s) of tasks/responsibilities
      This project is to be considered as a "proof of concept", aiming to support local actors (here in the Niayes area of Senegal) from a prospective questioning to the action by mobilizing the tools of anticipation and by coupling them with the ComMod approach.\\
      \textbf{My responsibility}: to guide the development of an experimental game to meet social objectives. Putting the game into production and acquiring the results.
    \end{cvitems}
  }

%---------------------------------------------------------
  \cventry
  {J.-P. Venot} % leaders
  {ANR DouBT} % project name
  {UMR G-EAU, Montpellier, FR} % Location
  {2018 -- 2019} % Date(s)
  {
    \begin{cvitems} % Description(s) of tasks/responsibilities
      The DouBT project (\emph{Deltas' Dealings with Uncertainty}), funded by the National Research Agency (France, ANR-15-ORAR-0002) proposes to provide innovative insights into ways of conceptualizing and managing uncertainty in the environmental planning sector of Southeast Asian deltas.\\
      \textbf{My responsibility}:Design of a game co-constructed with local actors and partners.  Supervision of workshops with stakeholders. Supervision of a Master 2 student on uncertainties.
    \end{cvitems}
  }

%---------------------------------------------------------
  \cventry
  {N. Gadet et A. Caron} % leaders
  {ProSuLi project} % project name
  {CIRAD, Montpellier, FR} % Location
  {2018 -- 2022} % Date(s)
  {
    \begin{cvitems} % Description(s) of tasks/responsibilities
      The European project ProSuLi (\emph{Promoting Sustainable Livelihoods in TFCAs}, Grant FED/2017/394-443) aims to explore the possibilities of collective management of \emph{TFCAs} (\emph{Transfrontier Conservation Areas}). A process of accompanying local stakeholders is being implemented in four sites (four southern African countries).\\
      \textbf{My responsibility} : for our Botswanan site, expertise and animation  of co-construction and prospective workshops. Elaboration of a game co-constructed with the actors. Facilitation of workshops and acquisition of results to help the decision of strategic orientations.
    \end{cvitems}
  }

%---------------------------------------------------------
  \cventry
  {D. Masse et A. Falot} % leaders
  {DSCATT project} % project name
  {CIRAD, Montpellier, FR} % Location
  {2018 -- 2023} % Date(s)
  {
    \begin{cvitems} % Description(s) of tasks/responsibilities
        DSCATT project (\emph{Dynamics of Soil Carbon Sequestration in Tropical and Temperate Agricultural systems}) is funded by the Agropolis Foundation and the Total Foundation. The interdisciplinary approaches aim to investigate the possibilities of mobilizing carbon storage in soils, through four sites (1 in France, 3 in Africa: Senegal, Zimbabwe, Kenya).\\
        \textbf{My responsibility}: leader of the WP3-Territory and co-leader of the WP4 "Companion modelling". Development of a methodology of knowledge explicitation. Supervision of two Master 2 students. Supervision of a post-doc.
    \end{cvitems}
  }

%---------------------------------------------------------
  \cventry
  {S. Aubert} % leaders
  {Task Force "Operationalizing the Commons".} % project name
  {CIRAD, Montpellier, FR} % Location
  {2018 -- 2021} % Date(s)
  {
    \begin{cvitems} % Description(s) of tasks/responsibilities
      This work is in line with the activities of the French Cooperation's "Land and Development" Technical Committee. The objective is to provide to the AFD (French Agency for Development) tools that will allow it to take into account the Commons and to promote their approach in development projects.\\
      \textbf{My responsibility} : expertise for the AFD, writing of a guide for development agents. And a policy brief for the French Cooperation's "Land and Development" Technical Committee
    \end{cvitems}
  }

%---------------------------------------------------------
  \cventry
  {D. Goffner} % leaders
  {ANR Future-Sahel} % project name
  {UMI 3189 CNRS, Dakar, SNG} % Location
  {2017 -- 2019} % Date(s)
  {
    \begin{cvitems} % Description(s) of tasks/responsibilities
      The Future-Sahel project, funded by the French National Research Agency (ANR-15-CE03-0001), aims to provide scientific and technical assistance to the National Agency of the Great Green Wall in Senegal.\\
      \textbf{My responsibility}: construction of a spatial database, guiding the development of methodologies for the quantification of tree-related uses in the Sahel, facilitation of workshops.
    \end{cvitems}
  }

%---------------------------------------------------------
  \cventry
  {M. Chevallier et E. Delay} % leaders
  {COMMONS Programm} % project name
  {UMR 6042 CNRS, Limoges, FR} % Location
  {2016} % Date(s)
  {
    \begin{cvitems} % Description(s) of tasks/responsibilities
      Le programme \textsc{Commons} "\textit{COllective ManageMent Of Natural reSources}" was financed by Limoges University International Call for Proposals (AOI) and by the University's Partnership Foundation. Objective: to structure a network of international partners to engage an interdisciplinary approach around the processes of collective management of natural resources.\\
      \textbf{My responsibility}: coordination of an international seminar around the questions of the Commons (Greece, Morocco, Portugal, France).
    \end{cvitems}
  }

%---------------------------------------------------------
  \cventry
  {N. Becu et M. Amalric} % leaders
  {LittoSim project} % project name
  {UMR 7266 CNRS, La Rochelle, FR} % Location
  {2015 -- 2018} % Date(s)
  {
    \begin{cvitems} % Description(s) of tasks/responsibilities
      The LittoSim project was financed within the framework of the "2015 Coastal Challenge" of the CNRS (French National Centre for Scientific Research) and brought together 11 French researchers. Objective: to develop a pedagogical tool to raise awareness of prevention and planning measures related to the risk of marine submersion.\\
      \textbf{My responsibility}: development of software bricks, debriefing methods and visualization tools.
    \end{cvitems}
      %\begin{cvsubentries}
      %  \cvsubentry{}{KNOX(Solution for Enterprise Mobile Security) Penetration Testing}{Sep. 2013}{}
      %  \cvsubentry{}{Smart TV Penetration Testing}{Mar. 2011 - Oct. 2011}{}
      %\end{cvsubentries}
    }

%---------------------------------------------------------

\cventry
{S. Leturcq et A. Lammoglia} % leaders
{VitiTerroir program} % project name
{UMR 7324 CNRS, Tours, FR} % Location
{2014 -- 2016} % Date(s)
{
  \begin{cvitems} % Description(s) of tasks/responsibilities
    The VitiTerroir program has developed a prospective tool based on a model of the transformations of wine-growing terroirs in the long term.\\
    \textbf{My responsibility} : co-animation of the spatial and temporal modeling axis of the project.
  \end{cvitems}
}

%---------------------------------------------------------
\cventry
{N. Ollat et J-M. Touzard} % leaders
{LACCAVE Program} % project name
{INRAE, FR} % Location
{2012 -- 2015} % Date(s)
{
  \begin{cvitems} % Description(s) of tasks/responsibilities
  LACCAVE (\emph{Long term impacts and Adaptation to Climate ChAnge in Viticulture and Enology}) brought together 23 research laboratories in France to assess the effects of climate change on grapevines and explore possible adaptation strategies. It continues today in LACCAVE 2.0.\\
  \textbf{My responsibility}: organization of a seminar on model coupling, organization of a study day Climate change and viticulture. Scientific animation.
  \end{cvitems}
}

%---------------------------------------------------------
\cventry
{M. Pontalti et F. Zottele} % leaders
{Ricerca sulla viticoltura di montagna} % project name
{Fondazione E. Mach, San Michele all'Adige, IT} % Location
{2011 -- 2018} % Date(s)
{
  \begin{cvitems} % Description(s) of tasks/responsibilities
    Studies and experiments for the maintenance of a mountain viticulture in Trentino and more broadly on European issues, in partnership with the CERVIM (Center of studies and research on mountain viticulture).\\
    \textbf{Ma responsabilité} : développement d’un \emph{workflow} de l’évaluation des paysages de terrasses et de viticulture dite « héroïque ».
  \end{cvitems}
}

%---------------------------------------------------------
\cventry
{H. Quénol} % leaders
{ANR TerViClim} % project name
{UMR 6554 CNRS, Rennes, FR} % Location
{2011 -- 2015} % Date(s)
{
  \begin{cvitems} % Description(s) of tasks/responsibilities
    The TerViClim project, for the observation and spatialization of the climate of the world's wine-growing terroirs in a context of climate change, financed by the French National Research Agency (ANR-07-JCJC-0069), has notably allowed the installation of temperature sensors in the Banyuls-Collioure AOC (Pyrénées-Orientales, FR) and in the Val di Cembra (Trentino, IT).\\
    \textbf{My responsibility} : spatial modelling of temperature dynamics at a fine scale in the vineyards of AOC Banyuls and Val di Cembra.
  \end{cvitems}
}

%---------------------------------------------------------

\end{cventries}
