%-------------------------------------------------------------------------------
%	SECTION TITLE
%-------------------------------------------------------------------------------
\cvsection{Accompagnement d'étudiants}

\cvsubsection{Co-encadrement de doctorant}
\vspace{2em}
%-------------------------------------------------------------------------------
%	CONTENT
%-------------------------------------------------------------------------------
\begin{cventries}
  
    %---------------------------------------------------------
    \cventry
      {François Vendel} % Affiliation/role
      {Modélisation et simulation autour de la grande muraille verte au Sénégal} % title de la thèse
      {MUSE, Montpellier, France} % Location
      {2022-2026} % Date(s)
      {
        \begin{cvitems} % Description(s) of experience/contributions/knowledge
          Modélisation d'accompagnement sur deux site d'intervention du projet de GMV au Sénégal.Thèse financé par le projet Dundi Ferlo.\\
          Directeur de Thèse : Pierre Bommel (CIRAD).\\
          Co-encadrement avec Jean-Daniel Cesaro.
        \end{cvitems}
      }
    %---------------------------------------------------------
    \cventry
      {Cheick A. Diloma Gabriel Traoré} % Affiliation/role
      {Couplage de modèles mathématiques et informatique de la transhumance sahélienne : cas du Sénégal} % title de la thèse
      {UCAD Dakar, SNG} % Location
      {2021-2023} % Date(s)
      {
        \begin{cvitems} % Description(s) of experience/contributions/knowledge
          Couplage de modèle à base d'agents et de théorie des graphes, pour chercher à optimiser les déplacement de pasteurs au Sénégal. Thèse financé par le projet CASSECs.\\
          Directeur de Thèse : Alassane Bah, professeur à l'école polytechnique de Dakar.\\
          Co-encadrement avec Camille Jahel (CIRAD), et Djibril Diop (ISRA BAME).)
        \end{cvitems}
      }
  
    %---------------------------------------------------------
    \cventry
      {Arthur Scriban} % Affiliation/role
      {SMA dans le bassin arachiser} % title de la thèse
      {MUSE, Montpellier, France} % Location
      {2021-2023} % Date(s)
      {
        \begin{cvitems} % Description(s) of experience/contributions/knowledge
         Impacte de l'intégration agriculture élevage dans le bassin arrachider au Sénégal. Approche de modélisation, simulation et exploration. Thèse financé par le projet DSCATT.\\
         Directeur de Thèse : Paulo Salgado (CIRAD).\\
         Co-encadrement avec Jonathan Vessière (CIRAD).
        \end{cvitems}
      }
    
    %---------------------------------------------------------
    \cventry
      {Gildas Assogba} % Affiliation/role
      {Entre jeux et modèle - propostion de couplage} % title de la thèse
      {Wageningen University \& Research (WUR)} % Location
      {2021-2023} % Date(s)
      {
        \begin{cvitems} % Description(s) of experience/contributions/knowledge
         Impacte de l'intégration agriculture élevage dans le bassin arrachider au Sénégal. Approche de modélisation, simulation et exploration. Thèse financé par le projet XXX.\\
         Directeur de Thèse : Katrien Descheemaeker, Wageningen University\\
         Co-encadrement Erika Speelman (WUR), Myriam Adam (CIRAD), David Berre (CIRAD).
        \end{cvitems}
      }  
    %---------------------------------------------------------
\end{cventries}

\vspace{2em}
\cvsubsection{Co-encadrement de Master}
\vspace{2em}
\begin{cventries}
%-------------------------------------------------------------------------------
%	CONTENT
%-------------------------------------------------------------------------------
    \cventry
        {Pape Ibrahima Thiam} % Affiliation/role
        {Surveillance et Détection des changements de zones à partir d’images satellites grâce aux modèles de Deep-Learning} % title
        {UCAD } % Location
        {février - Novembre 2024} % Date(s)
        {
        \begin{cvitems} % Description(s) of experience/contributions/knowledge
          Développer une méthodologie intégrant l’intelligence artificielle pour la surveillance et la détection des changements d'attribution du sol dans la région de Richard Toll, au Sénégal.\\
            Co-encadrement : Mandicou BA (UCAD), Idy Diop (UCAD), Alassane BAH (UCAD).
        \end{cvitems}
        }
        %--------------------------------------------------------- 
    \cventry
        {Rakya A. Ogueye} % Affiliation/role
        {Viabilité, SMA et One Health} % title
        {UCAD } % Location
        {Aout - février 2024} % Date(s)
        {
        \begin{cvitems} % Description(s) of experience/contributions/knowledge
            Réflexion sur une méthodologie partant d’un SMA comme modèle de simulation utilisé pour explorer de façon empirique le noyau de viabilité du système.\\
            Projet Santé et Territoire.\\
            Co-encadrement : Sophie Martin (INRAE), Isabelle Alvarez (INRAE), Raphaël Duboz (CIRAD).
        \end{cvitems}
        }
        %--------------------------------------------------------- 

    \cventry
        {Mathilde Hibon} % Affiliation/role
        {Entre Simulation participative et Simulation bayesienne. Quelle est le statuts du lieu pour les prospecteurs de lutte anti-achridienne Mauritanien} % title
        {Purpan, Toulouse} % Location
        {Févrer - Aout 2023} % Date(s)
        {
        \begin{cvitems} % Description(s) of experience/contributions/knowledge
          Diagnostic institutionnel et social de la zone du lac de Guiers. Mobilisation de la Grammaire institutionnelle pour évaluer et formaliser les relations sociales autour du lac. Couplage de l'\textit{IG} avec la grille de lecture TORSO. 
        \end{cvitems}
        }
        %---------------------------------------------------------
    \cventry
        {Lucas Broutin} % Affiliation/role
        {Entre Simulation participative et Simulation bayesienne. Quelle est le statuts du lieu pour les prospecteurs de lutte anti-achridienne Mauritanien} % title
        {Master 2 GAED, Nanterre, Paris, FR} % Location
        {Févrer - Aout 2023} % Date(s)
        {
        \begin{cvitems} % Description(s) of experience/contributions/knowledge
          Développement et test d'un prototype de simulation participative pour comprendre les mécanismes qui conduisent à l'intervention des prospecteurs dans la lutte contre le criquet pèlerin en Mauritanie. Relecture des résultats de simulation participative à l'aune des outils et concepts de la géographie.
        \end{cvitems}
        }
      %---------------------------------------------------------
    \cventry
        {Aboubacar GADIO} % Affiliation/role
        {Reflexion géographie pour une exploration d'accompagnement ; quelle concertation pour Diohine (Sénégal)} % title
        {Master 2, ENSA, Thiès, SNG} % Location
        {Aout - Fevrier 2022} % Date(s)
        {
        \begin{cvitems} % Description(s) of experience/contributions/knowledge
            Tester et implémenter une méthodologie permettant de produire des modèles mentaux des prospecteurs quand à une prise décision de traitement en Mauritanie.\\
            Projet AFD-CLCPRO
        \end{cvitems}
        }
    %---------------------------------------------------------

    \cventry
        {Fatoumata Sissoko} % Affiliation/role
        {Mémoire de fin de cycle d'ingénieur} % title
        {Institut polytechnique rural, Katibougou, ML} % Location
        {Aout - Février 2022} % Date(s)
        {
        \begin{cvitems} % Description(s) of experience/contributions/knowledge
            Tester et implémenter une méthodologie permettant de produire des modèles mentaux des prospecteurs quand à une prise décision de traitement en Mauritanie.\\
            Projet ADF-CLCPRO
        \end{cvitems}
        }
        %---------------------------------------------------------
    \cventry
        {Anna Ndiaye} % Affiliation/role
        {Mémoire de fin de cycle d'ingénieur} % title
        {M2, ENSA de Thiès} % Location
        {Févrer - Aout 2022} % Date(s)
        {
        \begin{cvitems} % Description(s) of experience/contributions/knowledge
            Mise en place et implémentation d'une méthodologie d'identification des pratiques agropastorales dans le territoire du Sénégal afin d’améliorer le bilan carbone (Commune de Téssékéré).\\
            Projet CASSECs
        \end{cvitems}
        }
        %---------------------------------------------------------
    \cventry
        {Lucas Broutin} % Affiliation/role
        {Reflexion géographie pour une exploration d'accompagnement ; quelle concertation pour Diohine (Sénégal)} % title
        {Master 2 GAED, Nanterre, Paris, FR} % Location
        {Févrer - Aout 2022} % Date(s)
        {
        \begin{cvitems} % Description(s) of experience/contributions/knowledge
            L'objectif est de mieux saisir les enjeux locaux gravitant autour de l’arbre (\textit{faidherbia albida}) et d'initier un processus concertatif autour de sa gestion. Cette concertation s'est fait autour d'un SMA co-construit et explorer avec les acteurs.\\
            Projet DSCATT
        \end{cvitems}
        }
        %---------------------------------------------------------
    \cventry
        {Oliver Foster} % Affiliation/role
        {Towards an Interdisciplinary Understanding of Carbon Sequestration in
        DSCATT} % title
        {Master 2 GAED, Nanterre, Paris, FR} % Location
        {Févrer - Aout 2020} % Date(s)
        {
        \begin{cvitems} % Description(s) of experience/contributions/knowledge
            L'objectif de cette étude est de formaliser et d'analyser les articulations possibles entre les compréhensions et les postures des chercheurs sur la séquestration du carbone à Murewa, dans le but plus large de contribuer à « ouvrir la boîte noire » (Latour, 1999) de la raison scientifique sur la séquestration du carbone dans les sols.\\
            Projet DSCATT, co-encadrement avec Arthur Perrotton, et Abigail Fallo
        \end{cvitems}
        }
    %---------------------------------------------------------
    \cventry
        {Yannick Diop} % Affiliation/role
        {Diversité des visions du monde et discours sur la séquestration du carbone dans les sols agricoles} % title
        {Master 2, SupAgro, Montpellier, FR} % Location
        {Févrer - Aout 2020} % Date(s)
        {
        \begin{cvitems} % Description(s) of experience/contributions/knowledge
            dentifier les différents points de vue et discours portés par les chercheurs sur la séquestration du carbone au niveau de la zone de Niakhar située au Sénégal, et explorer leurs articulations possibles.\\
            Projet DSCATT, co-encadrement avec Arthur Perrotton, et Abigail Fallot
        \end{cvitems}
        }
  

 
\end{cventries}
