%-------------------------------------------------------------------------------
%	SECTION TITLE
%-------------------------------------------------------------------------------
\cvsection{Expertises}


%-------------------------------------------------------------------------------
%	CONTENT
%-------------------------------------------------------------------------------
\begin{cventries}

  %---------------------------------------------------------
  \cventry
    {GRET} % Affiliation/role
    {Convention de programme - Communs 2} % title
    {Gorom-Lampsar, Niayes, SNG} % Location
    {2023} % Date(s)
    {
      \begin{cvitems} % Description(s) of experience/contributions/knowledge
        Cette expertise vise à accompagner les équipes du GRET dans la construction d'outils de formalisation des règles de gestion issu des approches ComMod.
      \end{cvitems}
    }

%---------------------------------------------------------
  \cventry
    {GRET} % Affiliation/role
    {Opportunité de l’approche par les Communs avec une entrée par les services} % title
    {Gorom-Lampsar, SNG} % Location
    {2021} % Date(s)
    {
      \begin{cvitems} % Description(s) of experience/contributions/knowledge
        Cette expertise a pour but d’évaluer l’opportunité de l’entrée par les services dans l’approche par les Communs. Je propose d’entrer par les réseaux de solidarités socio‑environnementales autour de la ressource. L’entrée par ces solidarités permettra d’identifier les situations d’actions dans lesquelles le service s’intègre de manière opportune.
      \end{cvitems}
    }

%---------------------------------------------------------
    \cventry
      {AFD} % Affiliation/role
      {Opérationnaliser l’approche par les Communs} % Organization/group
      {Chimanimani, MOZ} % Location
      {2019} % Date(s)
      {
        \begin{cvitems} % Description(s) of experience/contributions/knowledge
          Cette expertise, réalisée au Mozambique pour l'Agence française de développement (AFD, Paris), a pour but d'évaluer les opportunités d’utiliser la modélisation d’accompagnement pour expliciter les réseaux de solidarités socio‑environnementales aux acteurs locaux.
        \end{cvitems}
      }

%---------------------------------------------------------
  \cventry
    {AFD} % Affiliation/role
    {Opérationnaliser l’approche par les Communs} % Organization/group
    {Paris, FR} % Location
    {2019} % Date(s)
    {
      \begin{cvitems} % Description(s) of experience/contributions/knowledge
        Cette expertise, réalisée à l’Agence française de développement (AFD, Paris) a pour but de comprendre les temporalités des projets et de fournir un guide méthodologique aux agents de l’AFD. Ce guide identifie les marges de manœuvre d’une approche par les Communs dans le temps du projet.
      \end{cvitems}
    }

%---------------------------------------------------------
\end{cventries}
