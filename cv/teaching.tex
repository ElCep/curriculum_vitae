%-------------------------------------------------------------------------------
%	SECTION TITLE
%-------------------------------------------------------------------------------
\cvsection{Enseignements}


%-------------------------------------------------------------------------------
%	CONTENT
%-------------------------------------------------------------------------------
\begin{cventries}
  
  %---------------------------------------------------------
  \cventry
    {Chargé de cours} % Affiliation/role
    {Programme doctorale internationnal} % title
    {Sorbonne Université, FR} % Location
    {novembre 2022} % Date(s)
    {
      \begin{cvitems} % Description(s) of experience/contributions/knowledge
        16h de cours aux doctorants du programme international sur la modélisation d'accompagnement (ComMod) et la place de l'espace dans les systèmes complexes. 
      \end{cvitems}
    }

  %---------------------------------------------------------
  \cventry
    {Chargé de cours} % Affiliation/role
    {Master 2 Systèmes complexe} % title
    {\'Ecole Polytechnique Dakar, SNG} % Location
    {janvier - mars 2022} % Date(s)
    {
      \begin{cvitems} % Description(s) of experience/contributions/knowledge
      20h de cours de conceptions et réalisation automatique de cartes
      \end{cvitems}
    }

  %---------------------------------------------------------
  \cventry
    {Chargé de cours} % Affiliation/role
    {Master Politiques publiques} % title
    {Sciences Po Rennes, FR} % Location
    {mars 2021} % Date(s)
    {
      \begin{cvitems} % Description(s) of experience/contributions/knowledge
      2h de cours pour présenter l'hybridation entre les approches et les outils de l'anticipation et de ComMod : L'exemple des Niayes au Sénégal.
      \end{cvitems}
    }

  %---------------------------------------------------------
      \cventry
        {Key speaker} % Affiliation/role
        {Webinaire ComMod - GRET} % title
        {Paris, FR} % Location
        {2020} % Date(s)
        {
          \begin{cvitems} % Description(s) of experience/contributions/knowledge
            Proposer une lecture adaptée au GRET (Groupe de recherche et d’échange technologique, France) des implications socio‑techniques de la mobilisation de l’approche ComMod dans leurs projets, notamment dans le cadre de leurs travaux sur les Communs.
          \end{cvitems}
        }
        
  %---------------------------------------------------------
      \cventry
        {Co-organisateur} % Affiliation/role
        {Webinaire One Health dans les projets de développements} % title
        {Montpellier, FR} % Location
        {2020} % Date(s)
        {
          \begin{cvitems} % Description(s) of experience/contributions/knowledge
            Présenter l'intégration des approches de modélisation d'accompagnement avec les concepts portés par \textit{One health}
          \end{cvitems}
        }

  %---------------------------------------------------------
    \cventry
      {Co-organisateur} % Affiliation/role
      {Ecole thématique exModelo} % title
      {Châtenay sur Seine, FR} % Location
      {juin 2019 -- 2022} % Date(s)
      {
        \begin{cvitems} % Description(s) of experience/contributions/knowledge
          Ecole thématique sur l’exploration des modèles de simulation (analyse de sensibilité, calibration, validation, etc.) pour 25 participants. Elle s’adresse aux masters, doctorants·es, ingénieurs·es, chercheurs·euses académiques et aux entreprises qui s’intéressent à la modélisation, quel que soit leur domaine scientifique. L’objectif est d’apprendre aux participants et aux participantes à devenir autonomes dans l’exploration de leurs modèles.\\
          \url{https://exmodelo.org/}
          \end{cvitems}
      }
%---------------------------------------------------------
    \cventry
      {Co-organisateur} % Affiliation/role
      {Ecole thématique MISS-ABMS} % title
      {Montpellier, FR} % Location
      {septembre 2019, 2021} % Date(s)
      {
      \begin{cvitems} % Description(s) of experience/contributions/knowledge
        Ecole thématique pour 30 participants. Une semaine de formation à la modélisation multi-agents sur 3 plateforme : cormas, netlogo et gama.\\
        \url{https://www.agropolis.org/miss-abms/}
      \end{cvitems}
      }
%---------------------------------------------------------
    \cventry
      {Co-organisateur} % Affiliation/role
      {Ecole thématique ComMod} % title
      {Châteauneuf de Gadagne, FR} % Location
      {septembre 2019, 2020} % Date(s)
      {
      \begin{cvitems} % Description(s) of experience/contributions/knowledge
        Ecole thématique pour 30 participants. Une semaine de formation à la faciliation et à la modélisation d'accompagnement.\\
        \url{https://www.commod.org/}
      \end{cvitems}
      }
%---------------------------------------------------------
  \cventry
    {Co-organisateur} % Affiliation/role
    {Ecole thématique CNRS MAPS} % title
    {CAES CNRS - Olérons, FR} % Location
    {Juin 2017, 2018, 2019, 2021} % Date(s)
    {
      \begin{cvitems} % Description(s) of experience/contributions/knowledge
        Ecole thématique labellisée par le CNRS. Pour 30 participants, une semaine de formation au développement de modèles spatialisés de phénomènes socio-environnementaux. \url{https://maps.hypotheses.org/}
        \end{cvitems}
    }

%---------------------------------------------------------
\cventry
  {Chargé de cours} % Affiliation/role
  {Centre d'études doctorales IBN ZOHR} % title
  {Université IBN ZOHR Agadir, MA} % Location
  {Mars 2016} % Date(s)
  {
    \begin{cvitems} % Description(s) of experience/contributions/knowledge
      Co-animation d'une semaine de formation à la modélisation multi-agents à destination des doctorants de l'école doctorale IBN ZOHR à Agadir.
    \end{cvitems}
  }

%---------------------------------------------------------

\cventry
  {Chargé de cours} % Affiliation/role
  {Département de géographie} % title
  {Université de Limoges, FR} % Location
  {2013--2016} % Date(s)
  {
    \begin{cvitems} % Description(s) of experience/contributions/knowledge
    Chargé de 80 heures de TD en statistiques (avec R) pour les Licence 1 Géographie (40 h), et en géomatique (avec Qgis, GRASS‑GIS, R) pour les Master 2 Géographie/Histoire/Sociologie (20 h). J’ai également assuré deux modules (10 h chacun) à destination des Master 2 SHS et portant sur l’acquisition d’outils et méthodes de recherche innovants en sciences sociales : jeux sérieux et analyse de réseaux sociaux. A chaque fois, les cours étaient en présentiel et accompagnés d’un support électronique sur la plateforme d’e‑learning (moodle 2.X) de l’université.
    \end{cvitems}
  }

%---------------------------------------------------------

\end{cventries}
