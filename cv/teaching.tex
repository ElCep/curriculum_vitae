%-------------------------------------------------------------------------------
%	SECTION TITLE
%-------------------------------------------------------------------------------
\cvsection{Teachings}


%-------------------------------------------------------------------------------
%	CONTENT
%-------------------------------------------------------------------------------
\begin{cventries}
  %---------------------------------------------------------
  \cventry
    {Lecturer} % Affiliation/role
    {Master Politiques publiques} % title
    {Sciences Po Rennes, FR} % Location
    {March 2021} % Date(s)
    {
      \begin{cvitems} % Description(s) of experience/contributions/knowledge
      Present the hybridization between the approaches and tools of anticipation and ComMod: The example of the Niayes in Senegal.
      \end{cvitems}
    }
  %---------------------------------------------------------
      \cventry
        {Key speaker} % Affiliation/role
        {Webinaire ComMod - GRET} % title
        {Paris, FR} % Location
        {2020} % Date(s)
        {
          \begin{cvitems} % Description(s) of experience/contributions/knowledge
            To propose an adapted reading to GRET (Groupe de recherche et d'échange technologique, France) of the socio-technical implications of the mobilization of the ComMod approach in their projects, in particular in the framework of their work on the Commons.
          \end{cvitems}
        }
  %---------------------------------------------------------
      \cventry
        {Co-organizer} % Affiliation/role
        {Webinaire One Health dans les projets de développements} % title
        {Montpellier, FR} % Location
        {2020} % Date(s)
        {
          \begin{cvitems} % Description(s) of experience/contributions/knowledge
            To present the integration of companion modeling approaches with the concepts carried by \textit{One health}
          \end{cvitems}
        }

  %---------------------------------------------------------
    \cventry
      {Co-organizer} % Affiliation/role
      {Thematic school exModelo} % title
      {Châtenay sur Seine, FR} % Location
      {June 2019 -- 2021} % Date(s)
      {
        \begin{cvitems} % Description(s) of experience/contributions/knowledge
          Thematic school on the exploration of simulation models (sensitivity analysis, calibration, validation, etc.) for 25 participants. It is aimed at masters, PhD students, engineers, academic researchers and companies interested in modeling, whatever their scientific field. The objective is to teach participants to become autonomous in exploring their models.\\
          \url{https://exmodelo.org/}
          \end{cvitems}
      }
%---------------------------------------------------------
    \cventry
      {Co-organizer} % Affiliation/role
      {Thematic school MISS-ABMS} % title
      {Montpellier, FR} % Location
      {September 2019, 2021} % Date(s)
      {
      \begin{cvitems} % Description(s) of experience/contributions/knowledge
        Thematic school for 30 participants. Two weeks of training in multi-agent modeling on 3 platforms: cormas, netlogo and gama.\\
        \url{https://www.agropolis.org/miss-abms/}
      \end{cvitems}
      }
%---------------------------------------------------------
    \cventry
      {Co-organizer} % Affiliation/role
      {Thematic school ComMod} % title
      {Châteauneuf de Gadagne, FR} % Location
      {September 2019, 2020} % Date(s)
      {
      \begin{cvitems} % Description(s) of experience/contributions/knowledge
        Thematic school for 30 participants. A week of training in facilitation and Companion modelling.\\
        \url{https://www.commod.org/}
      \end{cvitems}
      }
%---------------------------------------------------------
  \cventry
    {Co-organizer} % Affiliation/role
    {Thematic school CNRS MAPS} % title
    {CAES CNRS - Olérons, FR} % Location
    {June 2017, 2018, 2019, 2021} % Date(s)
    {
      \begin{cvitems} % Description(s) of experience/contributions/knowledge
        Thematic school accredited by the CNRS. For 30 participants, a week of training in the development of spatialized models of socio-environmental phenomena. \url{https://maps.hypotheses.org/}
        \end{cvitems}
    }

%---------------------------------------------------------
\cventry
  {Lecturer} % Affiliation/role
  {Centre d'études doctorales IBN ZOHR} % title
  {Université IBN ZOHR Agadir, MA} % Location
  {March 2016} % Date(s)
  {
    \begin{cvitems} % Description(s) of experience/contributions/knowledge
      Co-animation of a training week on multi-agent modeling for PhD students of the IBN ZOHR PhD school in Agadir.
    \end{cvitems}
  }

%---------------------------------------------------------

\cventry
  {Lecturer} % Affiliation/role
  {Département de géographie} % title
  {Université de Limoges, FR} % Location
  {2013--2016} % Date(s)
  {
    \begin{cvitems} % Description(s) of experience/contributions/knowledge
    In charge of 80 hours of lectures in statistics (with R) for the Licence 1 Geography (40 h), and in geomatics (with Qgis, GRASS-GIS, R) for the Master 2 Geography/History/Sociology (20 h). I also taught two modules (10 h each) for Master 2 students on the acquisition of innovative research tools and methods in social sciences: serious games and social network analysis. Each time, the courses were in person and accompanied by an electronic support on the e-learning platform (moodle 2.X) of the university.
    \end{cvitems}
  }

%---------------------------------------------------------

\end{cventries}
